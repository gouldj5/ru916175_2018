% @Author: delengowski
% @Date:   2018-09-24 20:18:42
% @Last Modified by:   delengowski
% @Last Modified time: 2018-10-27 12:27:06
\documentclass[conference]{IEEEtran}
\usepackage{amsmath}
\usepackage{graphicx}
\usepackage{url}
\usepackage{mcode}
\usepackage{amsmath}
\usepackage{color}
\usepackage{xfrac}
%\usepackage{flushend}
\usepackage{float}
\usepackage{multicol}


\makeatletter
\renewcommand*\env@matrix[1][*\c@MaxMatrixCols c]{%
  \hskip -\arraycolsep
  \let\@ifnextchar\new@ifnextchar
  \array{#1}}
\makeatother

\begin{document}



  \title{Amplitude Modulation and Detection}
  \author{
    Matt Delengowski\\
    \IEEEauthorblockA{EComms Section 2\\
    \today\\
    Emails: delengowm1@students.rowan.edu}
  }
  \maketitle

  \IEEEpeerreviewmaketitle

  \begin{abstract}

  \end{abstract}

  \section{Introduction and Objectives}
  \label{sec:Intro}

  \section{Background and Relevant Theory}
  \label{sec:Background}


  \section{Procedure}
  \label{sec:Procedure}


  \section{Results and Discussion}
  \label{sec:Results}

    \subsection{Part 1}

      \subsubsection{Single-Tone AM Detection}

      \subsubsection{Multi-Tone AM Detection}

    \subsection{Part 2}

      \subsubsection{Single-Tone AM Detection}
        \begin{figure}[H]
            \includegraphics[scale=.5]{inv1layout.png}
            \caption{Circuit of CMOS Ring Oscillator}
            \label{fig:RingOsc}
        \end{figure}


  \section{Conclusions}
  \label{sec:Conclusions}


  \begin{thebibliography}{99}

  \bibitem{Book:VLSI}
  L. W. Couch, Digital and analog communication systems, 8th ed. Upper Saddle River, N.J: Prentice Hall, 2013.

  \end{thebibliography}

  \vfill
  \columnbreak
  \newpage
  \section{Appendix}
  \begin{verbatim}


  \end{verbatim}



\end{document}
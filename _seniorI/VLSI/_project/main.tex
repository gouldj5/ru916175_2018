\documentclass[conference]{IEEEtran}

\usepackage{amsmath}
\usepackage{graphicx}
\usepackage{url}
\usepackage{mcode}
\usepackage{color}
\usepackage{float}
\usepackage{xfrac}
\usepackage{fancyvrb}
\usepackage{tabu}

%
\renewcommand*\env@matrix[1][*\c@MaxMatrixCols c]{%
  \hskip -\arraycolsep
  \let\@ifnextchar\new@ifnextchar
  \array{#1}}
\makeatother

\begin{document}



\title{Lab 5: Extraction and Layout Versus Schematic }
\author{
Joshua Gould\\
\IEEEauthorblockA{VLSI 09414\\
Date: 10-26-2018\\
Emails: gouldj5@students.rowan.edu}
}
\maketitle

\IEEEpeerreviewmaketitle

\begin{abstract}
The circuit design layout design of CMOS logic circuits determines a compact architecture for physically designing a logical transistor circuit. Following design rules for NOT, NAND, and NOR logic schematics allowed to design the fabrication using the smallest area possible. Circuit layout design also displays differences in logic overview and characteristics of layer distribution. Visually, it is simple to tell that NOR logic layouts require more surface area than the NAND logic circuits. Equivocally, NAND logic design is more cost-effective and possesses a higher storage capacity in logic design for its physical layout size.
\end{abstract}


%\VerbatimInput{netlist.txt}
\end{document}
